\justify

Our project would consist of a high-performance seed-and-extend $x$-drop alignment implementation which would benefit both BELLA and many other computational biology applications.
The project would exploit two different levels of parallelism:
\begin{itemize}
	\item Intra-pair parallelism;
	\item Inter-pair parallelism.
\end{itemize}

The first component of our project would aim at improving performance at single pairwise alignment level exploiting instruction level parallelism. 	
Our implementation would initially target AVX2 extensions and it would result in a \textit{banded} $x$-drop alignment.
The vector width would dictate the band width of the alignment.
The width of the SIMD register file for AVX2 is 256 bits.
Thus, assuming our scoring system uses integer values at 16 bits, we would be able to fit 16 values in a vector, obtaining a band width of 16. 
Each anti-diagonal in the matrix would be computed in parallel through the vectorization, after which we would check the termination condition ($x$-drop) and update the alignment extension using vector masking. 	
Once implemented a first working version using AVX2 extension, we would consider to implement it also for AVX512 extension, resulting in a doubled band width of the alignment.
%A further implementation might consist in removing the band limit. 	

The second component would instead use multi-core parallelism resulting in multiple seed-and-extend $x$-drop pairwise alignments occurring in parallel. This might result in either a one-to-many or many-to-many implementation.
A one-to-many implementation might be more convenient for BELLA as, given its overlap matrix before the alignment, each column is a sequence and each non-zero of that column is an overlapping sequence with one or two common seeds.
For both components, we might consider to exploit an intrinsic parallelism of the seed-and-extend paradigm, that is computing the left and right alignment extensions simultaneously. This might be achieved exploiting hyper-threading, and properly pinning threads to cores.
Finally, we would merge these two levels of parallelism in a single multi-core SIMD implementation.

%* Roofline model for the SIMD version.
%* How is data reuse in multi-core implementation.
